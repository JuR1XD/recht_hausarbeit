% !TEX root =  master.tex
\chapter{Zusammenfassung}

Im Laufe der Arbeit wurde deutlich, dass das Handelsrecht im HGB eine bedeutende Rolle für die Regelung und den Schutz von Geschäften im Handelsverkehr spielt.

Persönlich finde ich diese Themen äußerst relevant und praxisnah. Das Verständnis der einzelnen Handelsgeschäfte nach dem HGB ermöglicht es den Akteuren im Handelsverkehr, ihre Rechte und Pflichten besser zu verstehen und angemessen zu handeln. Die Regelungen zum Speditions- und Frachtvertrag bieten eine klare rechtliche Grundlage für den Transport von Gütern und stellen sicher, dass sowohl der Spediteur als auch der Frachtführer bestimmten Haftungsbestimmungen unterliegen.

Die intensive Auseinandersetzung mit diesen Themen hat mir ein besseres Verständnis für die rechtlichen Aspekte des Handelsverkehrs vermittelt. Die Kenntnis dieser rechtlichen Bestimmungen ist von entscheidender Bedeutung, um potenzielle Risiken zu minimieren und rechtliche Konflikte zu vermeiden.

Die Hausarbeit hat mir gezeigt, dass das HGB ein unverzichtbares Instrument ist, um den Handelsverkehr zu regeln und einen reibungslosen Ablauf von Handelsgeschäften zu gewährleisten. Außerdem war es ebenfalls eine große Hilfe bei der korrekten Recherche in dieser Hausarbeit. Es ist von großer Bedeutung, dass Unternehmen und Einzelpersonen, die im Handelsverkehr tätig sind, sich mit den Bestimmungen des HGB vertraut machen und ihre Handelsaktivitäten entsprechend gestalten.

Insgesamt bin ich der Überzeugung, dass die Themen der einzelnen Handelsgeschäfte, des Speditions- und Frachtvertrags sowie der Lager- und Transportgeschäfte nach dem HGB eine solide Grundlage für den Handelsverkehr bilden. Sie bieten Rechtssicherheit, ermöglichen effiziente Geschäftsabläufe und schützen die Interessen der Vertragsparteien. Daher sollten Unternehmen und Einzelpersonen, die im Handelsverkehr tätig sind, sich intensiv mit diesen Themen auseinandersetzen, um von den rechtlichen Vorteilen und Schutzmechanismen des HGB zu profitieren.


