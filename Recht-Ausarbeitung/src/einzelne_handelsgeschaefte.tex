% !TEX root =  master.tex

\chapter[Einzelne Handelsgeschäfte]{Einzelne Handelsgeschäfte \footnote{Quelle:~\cite{Handelsrecht} \&~\cite{Großkommentar_HGB}}}

\section{Definition Handelskauf}
Der Handelskauf gemäß dem HGB ist eine besondere Form des Kaufvertrags, bei dem mindestens eine der beteiligten Parteien ein Kaufmann ist oder bei dem es sich um einen beiderseitigen Handelskauf handelt.

Ein Handelskauf liegt vor, wenn eine Vereinbarung über den Kauf von Waren getroffen wird. Dabei umfasst der Begriff "Waren" bewegliche Sachen, also Güter, die körperlich greifbar sind.


\section{Handelsgeschäft und beiderseitiges Handelsgeschäft}
Die Unterscheidung zwischen einem Handelsgeschäft und einem beiderseitigen Handelsgeschäft ist von Bedeutung, um die Anwendung spezifischer rechtlicher Vorschriften im Rahmen des Handelsrechts zu bestimmen. Gemäß § 1 Abs. 1 HGB wird ein Handelsgeschäft definiert als ein Geschäft, an dem mindestens eine der beteiligten Parteien ein Kaufmann ist. Ein Kaufmann ist eine natürliche oder juristische Person, die ein Handelsgewerbe betreibt.

Unter einem beiderseitigen Handelsgeschäft nach §§ 377-379 HGB hingegen handelt es sich um ein Geschäft, bei dem sowohl der Verkäufer als auch der Käufer Kaufleute sind. Das bedeutet, dass beide Parteien im Handelsregister eingetragen sind oder aufgrund anderer Kriterien als Kaufleute gelten. In solchen Fällen werden spezielle Vorschriften im HGB angewendet, die ausschließlich für beiderseitige Handelsgeschäfte gelten.

Die Unterscheidung zwischen einem Handelsgeschäft und einem beiderseitigen Handelsgeschäft ist wichtig, da sie die Rechtsstellung und die Pflichten der beteiligten Parteien beeinflusst. Im Falle eines beiderseitigen Handelsgeschäfts sind beide Parteien Kaufleute und unterliegen den besonderen Bestimmungen des HGB, die beispielsweise Regelungen zur Handelskaufmannschaft, Handelsbüchern und Handelsregistern enthalten. Für andere Handelsgeschäfte, bei denen nur eine Partei Kaufmann ist, gelten die allgemeinen Regelungen des HGB, die für sämtliche Handelsgeschäfte Anwendung finden.
\section{Anforderungen an einen Handelskauf}
Damit ein Handelskauf nach dem HGB vorliegt, ist es erforderlich, dass zumindest eine der beteiligten Parteien als Kaufmann im Sinne des HGB gilt. Die Kaufmannseigenschaft kann sich auf verschiedene Weisen ergeben und ist eng mit der gewerblichen Tätigkeit einer Person verbunden. Ein Kaufmann ist in der Regel eine natürliche oder juristische Person, die ein Handelsgewerbe betreibt.

Die Kaufmannseigenschaft kann aufgrund verschiedener Merkmale festgestellt werden. Eines dieser Merkmale ist beispielsweise die Eintragung im Handelsregister. Durch die Registrierung werden Kaufleute öffentlich bekannt gemacht und ihre Handelstätigkeiten werden transparent. Eine weitere Möglichkeit, die Kaufmannseigenschaft zu erlangen, besteht in der Mitgliedschaft in einer Handelskammer oder einer vergleichbaren Organisation, die den Status eines Kaufmanns bestätigt.

Die Kaufmannseigenschaft ist von großer Bedeutung, da sie Auswirkungen auf die rechtlichen Rahmenbedingungen des Handelskaufs hat. Das HGB enthält spezifische Vorschriften und Regelungen, die ausschließlich für Kaufleute gelten. Diese handelsrechtlichen Bestimmungen beinhalten beispielsweise Regelungen zur Buchführung, Bilanzierung, zum Handelsregister und zur Handelsfirma. Sie dienen der Schaffung von Rechtssicherheit und gewährleisten einheitliche Standards im Handelsverkehr.
\newpage\section{Geltungsbereich des Handelskaufs und Ausnahmen}
Der Geltungsbereich des Handelskaufs erstreckt sich ausschließlich auf den Erwerb von beweglichen Sachen, die als "Waren" bezeichnet werden. Diese Waren sind physische Objekte, die man anfassen und transportieren kann, wie beispielsweise Kleidung, Elektronikgeräte oder Maschinen. Im Handelskauf werden also spezifische Regelungen und Vorschriften für den Handel mit solchen Gegenständen festgelegt.

Es ist wichtig zu beachten, dass der Handelskauf keine Anwendung auf den Kauf von Grundstücken, Rechten oder Forderungen findet. Diese Vermögenswerte gelten nicht als Waren im Sinne des Handelskaufs. Der Handelskauf bezieht sich vielmehr auf den Austausch von materiellen Gütern. Dadurch wird sichergestellt, dass die spezifischen Regelungen des Handelsrechts nur für den Handel mit beweglichen Sachen gelten, während andere Rechtsbereiche für den Erwerb von Grundstücken oder immateriellen Vermögenswerten relevant sind. Somit sind die Bestimmungen des Handelskaufs nicht auf den Erwerb von Grundstücken oder anderen immateriellen Vermögenswerten anwendbar.
