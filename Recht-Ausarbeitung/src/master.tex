%%%%%%%%%%%%%%%%%%%%%%%%%%%%%%%%%%%%%%%%%%%%%%%%%%%%%%%%%%
%   Autoren:
%   Prof. Dr. Bernhard Drabant
%   Prof. Dr. Dennis Pfisterer
%   Prof. Dr. Julian Reichwald
%%%%%%%%%%%%%%%%%%%%%%%%%%%%%%%%%%%%%%%%%%%%%%%%%%%%%%%%%%

%%%%%%%%%%%%%%%%%%%%%%%%%%%%%%%%%%%%%%%%%%%%%%%%%%%%%%%%%%
%	ANLEITUNG: 
%   1. Ersetzen Sie da Bild firmenlogo im Verzeichnis img
%   2. Passen Sie alle Stellen im Dokument an, die mit
%      @stud
%      markiert sind
%%%%%%%%%%%%%%%%%%%%%%%%%%%%%%%%%%%%%%%%%%%%%%%%%%%%%%%%%%

%%%%%%%%%%%%%%%%%%%%%%%%%%%%%%%%%%%%%%%%%%%%%%%%%%%%%%%%%%
%	ACHTUNG:
%   Für das Erstellen des Literaturverzeichnisses wird das
%   modernere Paket biblatex in Kombination mit biber
%   verwendet - nicht mehr das ältere Paket BibTex!
%
%   Bitte stellen Sie Ihre TeX-Umgebung entsprechend ein (z.B. TeXStudio):
%   Einstellungen --> Erzeugen --> Standard Bibliographieprogramm: biber
%%%%%%%%%%%%%%%%%%%%%%%%%%%%%%%%%%%%%%%%%%%%%%%%%%%%%%%%%%

\documentclass[fontsize=12pt,BCOR=5mm,DIV=12,parskip=half,listof=totoc,
    paper=a4,toc=bibliography,pointlessnumbers]{scrreprt}

%toc=listof,listof=entryprefix,

\makeindex

%% Elementare Pakete, Konfigurationen und Definitionen werden geladen (gegebenenfalls anpassen)
\input{config}

%%
%% @stud
%%
%% PERSÖNLICHE ANGABEN (BITTE VOLLSTÄNDIG EINGEBEN zwischen den Klammern: {...})
%%
\ArtDerArbeit{Haus}
\TitelDerArbeit{Einzelne Handelsgeschäfte\nl Fracht- und Speditionsvertrag\nl Lager- und Transportgeschäfte}
\AutorDerArbeit{Juri Hofmann}
\Firma{ING DiBa AG}
\Kurs{WWI22SEB}
\Studienrichtung{Software Engineering}
\Matrikelnummer{4850943}
\Studiengangsleiter{Prof. Dr. Thomas Holey}
\Bearbeitungszeitraum{14.06.2023 -- 21.06.2023}

%%
%% @stud
%%
%% BIBLIOGRAPHY (@stud: Bibliographie-Stil wählen - Position und Indizierung)
%%  Auswahl zwischen:
%%   NUMERIC Style
%%   IEEE Style
%%   ALPHABETIC Style
%%   HARVARD Style
%%   CHICAGO Style
%%   (oder eigenen zulässigen Stil wählen)
%%
%%%%%%%%%%%%%
%% Zitierstil
%%%%%%%%%%%%%
% NUMERIC Style - e. g. [12]
\newcommand{\indextype}{numeric}
%
% IEEE Style - numeric kind of style
%\newcommand{\indextype}{ieee}
%
% ALPHABETIC Style - e. g. [AB12]
%\newcommand{\indextype}{alphabetic}
%
% HARVARD Style
%\newcommand{\indextype}{apa}
%
% CHICAGO Style
%\newcommand{\indextype}{authoryear}
%
%%%%%%%%%%%%%%%%%%%%%%
%% Position des Zitats
%%%%%%%%%%%%%%%%%%%%%%
\newcommand{\position}{inline}
%
% (!!) FOOTNOTE POSITION NOT RECOMMENDED IN MINT DOMAIN:
%\newcommand{\position}{footnote}

%% Final: Setzen des Zitierstils und der Zitatposition
\usepackage[backend=biber, autocite=\position, style=\indextype]{biblatex}
\settingBibFootnoteCite

%%
%% Definitionen und Commands
%%
\newcommand{\abs}{\par\vskip 0.2cm\goodbreak\noindent}
\newcommand{\nl}{\par\noindent}
\newcommand{\mcl}[1]{\mathcal{#1}}
\newcommand{\nowrite}[1]{}
\newcommand{\NN}{{\mathbb N}}
\newcommand{\imagedir}{img}

\makeindex

\begin{document}

    \setTitlepage
%%%%%%%%%%%%%%%%%%%%%%%%%%%%%%%%%%%
% VERZEICHNISSE und ABSTRACT
%
% @stud: ggf. nicht benötigte Verzeichnisse auskommentieren/löschen
%
    \tableofcontents
    \cleardoublepage

% Abkürzungsverzeichnis
% @stud: acronyms.tex bearbeiten
    % !TEX root =  master.tex
\clearpage
\chapter*{Abkürzungsverzeichnis}
\addcontentsline{toc}{chapter}{Abkürzungsverzeichnis}

\begin{acronym}[XXXXXXX]
	\acro{DHBW}{Duale Hochschule Baden-Württemberg}
	\acro{HGB}{Handelsgesetzbuch}
	\acro{ADSp}{Allgemeine Deutsche Spediteurbedingungen}
\end{acronym}
    \cleardoublepage

%%%%%%%%%%%%%%%%%%%%%%%%%%%%%%%%%%%%%%%%%%%%%%%%%%%%%%%%%%%%%%%%%%%%%%%%%%%%%%%%%%%%%%%%%%
% KAPITEL UND ANHÄNGE
%
% @stud:
%   - nicht benötigte: auskommentieren/löschen
%   - neue: bei Bedarf hinzufügen mittels input-Kommando an entsprechender Stelle einfügen
%%%%%%%%%%%%%%%%%%%%%%%%%%%%%%%%%%%%%%%%%%%%%%%%%%%%%%%%%%%%%%%%%%%%%%%%%%%%%%%%%%%%%%%%%%

    \initializeText
    \onehalfspacing

%%%%%%%%%%%%%%%%%%%%%%%%%%%%%%%%%%%
% KAPITEL
%
% @stud: einzelne Kapitel bearbeiten und eigene Kapitel hier einfügen
%%%%%%%%%%%%%%%%%%%%%%%%%%%%%%%%%%%
    % !TEX root =  master.tex

\chapter[Einzelne Handelsgeschäfte]{Einzelne Handelsgeschäfte \footnote{Quelle:~\cite{Handelsrecht} \&~\cite{Großkommentar_HGB}}}

\section{Definition Handelskauf}
Der Handelskauf gemäß dem HGB ist eine besondere Form des Kaufvertrags, bei dem mindestens eine der beteiligten Parteien ein Kaufmann ist oder bei dem es sich um einen beiderseitigen Handelskauf handelt.

Ein Handelskauf liegt vor, wenn eine Vereinbarung über den Kauf von Waren getroffen wird. Dabei umfasst der Begriff "Waren" bewegliche Sachen, also Güter, die körperlich greifbar sind.


\section{Handelsgeschäft und beiderseitiges Handelsgeschäft}
Die Unterscheidung zwischen einem Handelsgeschäft und einem beiderseitigen Handelsgeschäft ist von Bedeutung, um die Anwendung spezifischer rechtlicher Vorschriften im Rahmen des Handelsrechts zu bestimmen. Gemäß § 1 Abs. 1 HGB wird ein Handelsgeschäft definiert als ein Geschäft, an dem mindestens eine der beteiligten Parteien ein Kaufmann ist. Ein Kaufmann ist eine natürliche oder juristische Person, die ein Handelsgewerbe betreibt.

Unter einem beiderseitigen Handelsgeschäft nach §§ 377-379 HGB hingegen handelt es sich um ein Geschäft, bei dem sowohl der Verkäufer als auch der Käufer Kaufleute sind. Das bedeutet, dass beide Parteien im Handelsregister eingetragen sind oder aufgrund anderer Kriterien als Kaufleute gelten. In solchen Fällen werden spezielle Vorschriften im HGB angewendet, die ausschließlich für beiderseitige Handelsgeschäfte gelten.

Die Unterscheidung zwischen einem Handelsgeschäft und einem beiderseitigen Handelsgeschäft ist wichtig, da sie die Rechtsstellung und die Pflichten der beteiligten Parteien beeinflusst. Im Falle eines beiderseitigen Handelsgeschäfts sind beide Parteien Kaufleute und unterliegen den besonderen Bestimmungen des HGB, die beispielsweise Regelungen zur Handelskaufmannschaft, Handelsbüchern und Handelsregistern enthalten. Für andere Handelsgeschäfte, bei denen nur eine Partei Kaufmann ist, gelten die allgemeinen Regelungen des HGB, die für sämtliche Handelsgeschäfte Anwendung finden.
\section{Anforderungen an einen Handelskauf}
Damit ein Handelskauf nach dem HGB vorliegt, ist es erforderlich, dass zumindest eine der beteiligten Parteien als Kaufmann im Sinne des HGB gilt. Die Kaufmannseigenschaft kann sich auf verschiedene Weisen ergeben und ist eng mit der gewerblichen Tätigkeit einer Person verbunden. Ein Kaufmann ist in der Regel eine natürliche oder juristische Person, die ein Handelsgewerbe betreibt.

Die Kaufmannseigenschaft kann aufgrund verschiedener Merkmale festgestellt werden. Eines dieser Merkmale ist beispielsweise die Eintragung im Handelsregister. Durch die Registrierung werden Kaufleute öffentlich bekannt gemacht und ihre Handelstätigkeiten werden transparent. Eine weitere Möglichkeit, die Kaufmannseigenschaft zu erlangen, besteht in der Mitgliedschaft in einer Handelskammer oder einer vergleichbaren Organisation, die den Status eines Kaufmanns bestätigt.

Die Kaufmannseigenschaft ist von großer Bedeutung, da sie Auswirkungen auf die rechtlichen Rahmenbedingungen des Handelskaufs hat. Das HGB enthält spezifische Vorschriften und Regelungen, die ausschließlich für Kaufleute gelten. Diese handelsrechtlichen Bestimmungen beinhalten beispielsweise Regelungen zur Buchführung, Bilanzierung, zum Handelsregister und zur Handelsfirma. Sie dienen der Schaffung von Rechtssicherheit und gewährleisten einheitliche Standards im Handelsverkehr.
\newpage\section{Geltungsbereich des Handelskaufs und Ausnahmen}
Der Geltungsbereich des Handelskaufs erstreckt sich ausschließlich auf den Erwerb von beweglichen Sachen, die als "Waren" bezeichnet werden. Diese Waren sind physische Objekte, die man anfassen und transportieren kann, wie beispielsweise Kleidung, Elektronikgeräte oder Maschinen. Im Handelskauf werden also spezifische Regelungen und Vorschriften für den Handel mit solchen Gegenständen festgelegt.

Es ist wichtig zu beachten, dass der Handelskauf keine Anwendung auf den Kauf von Grundstücken, Rechten oder Forderungen findet. Diese Vermögenswerte gelten nicht als Waren im Sinne des Handelskaufs. Der Handelskauf bezieht sich vielmehr auf den Austausch von materiellen Gütern. Dadurch wird sichergestellt, dass die spezifischen Regelungen des Handelsrechts nur für den Handel mit beweglichen Sachen gelten, während andere Rechtsbereiche für den Erwerb von Grundstücken oder immateriellen Vermögenswerten relevant sind. Somit sind die Bestimmungen des Handelskaufs nicht auf den Erwerb von Grundstücken oder anderen immateriellen Vermögenswerten anwendbar.

    %! Author = juri
%! Date = 13.06.23

% Preamble
\documentclass[11pt]{article}


% Document
\begin{document}

    \chapter[Fracht- und Speditionsvertrag]{Fracht- und Speditionsvertrag}

    \section[]{}

\end{document}
    % !TEX root =  master.tex


\chapter[Lagergeschäfte]{Lagergeschäfte}

\section{Definition}
Lagergeschäfte im Sinne des HGB beziehen sich auf eine spezielle Art von entgeltlichen Verträgen, die zwischen einem Lagerhalter und einem Einlagerer abgeschlossen werden. Diese Verträge regeln die Verwahrung und Aufbewahrung von Waren durch den Lagerhalter.

Gemäß § 467 HGB liegt ein Lagergeschäft vor, wenn der Lagerhalter die Verantwortung für die ordnungsgemäße Aufbewahrung, Pflege und Sicherheit der eingelagerten Güter übernimmt. Dabei handelt es sich um eine Dienstleistung, bei der der Lagerhalter gegen Entgelt die Waren des Einlagerers in seinem Lager aufnimmt und für deren Verwahrung sorgt.

Ein Lagergeschäft kann verschiedene Formen annehmen. Es kann sich beispielsweise um die Lagerung von Rohstoffen, Halbfertigprodukten oder Endprodukten handeln. Die Waren können sowohl unbewegliche als auch bewegliche Güter umfassen. Typische Beispiele für Lagergeschäfte sind die Lagerung von Waren in Lagerräumen, Lagerhallen oder anderen Lagerstätten.

Die Definition von Lagergeschäften im HGB legt den Fokus auf die rechtliche und vertragliche Beziehung zwischen dem Lagerhalter und dem Einlagerer. Der Lagerhalter übernimmt die Verantwortung für die ordnungsgemäße Verwahrung der Waren und hat die Pflicht, diese vor Schäden oder Verlusten zu schützen. Der Einlagerer hingegen vertraut dem Lagerhalter die sichere Aufbewahrung seiner Waren an.
\section{Rechte und Pflichten des Lagerhalters}

Die Rechte und Pflichten des Lagerhalters sind im HGB geregelt und beziehen sich auf seine Verantwortung für die ordnungsgemäße Verwahrung und Aufbewahrung der eingelagerten Waren. Als Lagerhalter übernimmt er eine wichtige Rolle und hat verschiedene Aufgaben und Verpflichtungen zu erfüllen.
\begin{itemize}
    \item \textbf{Aufbewahrungspflicht:} Der Lagerhalter ist verpflichtet, die ihm anvertrauten Waren sorgfältig und ordnungsgemäß aufzubewahren. Er muss dafür sorgen, dass die Waren vor Schäden, Verlusten oder Diebstahl geschützt sind.
    \item \textbf{Bestandsaufnahme und Dokumentation:} Der Lagerhalter hat die Pflicht, eine genaue Bestandsaufnahme der eingelagerten Waren vorzunehmen und diese zu dokumentieren. Dies beinhaltet die Erfassung von Art, Menge, Zustand und Wert der Waren.
    \item \textbf{Herausgabepflicht:} Der Lagerhalter ist verpflichtet, die eingelagerten Waren gemäß den Anweisungen des Einlagerers herauszugeben. Der Lagerhalter muss die Waren ordnungsgemäß verpackt und unversehrt übergeben.
    \item \textbf{Vergütung:} Der Lagerhalter hat Anspruch auf eine angemessene Vergütung für seine Lagerdienstleistung.
    \item \textbf{Pfandrecht:} Der Lagerhalter besitzt ein gesetzliches Pfandrecht an den eingelagerten Waren. Dies bedeutet, dass er unter bestimmten Voraussetzungen ein Zurückbehaltungsrecht ausüben kann, um offene Forderungen gegenüber dem Einlagerer zu sichern. Das Pfandrecht ermöglicht es dem Lagerhalter, die Herausgabe der Waren zu verweigern, bis die offenen Zahlungen beglichen sind.
\end{itemize}

\section{Haftung des Lagerhalters für Schäden oder Verluste}
Die Haftung des Lagerhalters ist ein wesentlicher Aspekt im Zusammenhang mit Lagergeschäften. Sie bezieht sich auf die Verantwortung des Lagerhalters für Schäden oder Verluste, die während der Lagerung der Waren auftreten können. Im HGB sind spezifische Bestimmungen zur Haftung des Lagerhalters festgelegt, die sowohl seine Rechte als auch seine Pflichten regeln.

\begin{itemize}
    \item \textbf{Haftungsumfang:} Der Lagerhalter haftet für Schäden oder Verluste der Waren, außer er kann nachweisen, dass er seine Sorgfaltspflicht erfüllt hat und der Schaden unabhängig von seinem Verschulden eingetreten ist.
    \item \textbf{Sorgfaltspflicht:} Der Lagerhalter muss die Waren sorgfältig behandeln, indem er angemessene Lagerbedingungen sicherstellt und regelmäßige Kontrollen durchführt.
    \item \textbf{Beschränkung der Haftung:} Der Lagerhalter kann seine Haftung im Lagervertrag beschränken oder ausschließen, sofern dies klar und eindeutig vereinbart wird und keine gesetzlichen Bestimmungen verletzt werden.
    \item \textbf{Versicherung:} Der Lagerhalter kann eine Versicherung abschließen, um die Haftung zu begrenzen und mögliche Risiken abzudecken.
    \item \textbf{Ausnahme von der Haftung:} Der Lagerhalter kann von seiner Haftung befreit sein, wenn der Schaden oder Verlust auf Umständen höherer Gewalt beruht.
    \item \textbf{Schadensersatzansprüche:} Einlagerer haben das Recht auf Schadensersatz, der den Wert der beschädigten oder verlorenen Waren sowie mögliche Folgeschäden umfasst.
\end{itemize}

Die Rechte und Pflichten des Lagerhalters sind im HGB geregelt und beziehen sich auf seine Verantwortung für die ordnungsgemäße Verwahrung und Aufbewahrung der eingelagerten Waren. Als Lagerhalter übernimmt er eine wichtige Rolle und hat verschiedene Aufgaben und Verpflichtungen zu erfüllen.

\section{Kündigung und Beendigung des Lagervertrags}
Die Kündigung und Beendigung des Lagervertrags ist ein wichtiger Aspekt im Bereich der Lagergeschäfte. Eine korrekte Kündigung und Beendigung des Lagervertrags gemäß den Bedingungen und gesetzlichen Vorschriften ist wichtig, um einen reibungslosen Abschluss der Lagervereinbarung zu gewährleisten.

\begin{itemize}
    \item \textbf{Kündigung durch den Lagerhalter:} Der Lagerhalter kann den Lagervertrag gemäß den vereinbarten Bedingungen und Fristen schriftlich kündigen.
    \item \textbf{Kündigung durch den Einlagerer:} Der Einlagerer kann den Lagervertrag ebenfalls schriftlich kündigen, unter Beachtung der vereinbarten Kündigungsbedingungen.
    \item \textbf{Beendigung durch Erfüllung des Vertrags:} Der Lagervertrag endet automatisch, wenn die vereinbarte Lagerdauer abgelaufen ist und die Waren ordnungsgemäß zurückgegeben wurden.
    \item \textbf{Außerordentliche Kündigung:} In Ausnahmefällen kann eine außerordentliche Kündigung gerechtfertigt sein, wenn eine Vertragspartei wesentliche Pflichten verletzt oder eine Fortsetzung unzumutbar wäre.
    \item \textbf{Abwicklung der Beendigung:} Bei Vertragsbeendigung müssen sowohl der Lagerhalter als auch der Einlagerer ihre Verpflichtungen erfüllen, wie die ordnungsgemäße Warenübergabe und Zahlung offener Beträge.
\end{itemize}





\chapter[Transportgeschäfte]{Transportgeschäfte}

\section{Definition}

Transportgeschäfte sind Verträge, bei denen der Transport von Gütern gegen Entgelt erfolgt. Sie umfassen den Land-, Wasser- und Lufttransport, können national oder international sein und werden von Frachtführern, Speditionen oder Logistikunternehmen durchgeführt. Der Kern eines Transportgeschäfts besteht aus der Vereinbarung zwischen Absender und Frachtführer über den Ort, die Zeit und die Bedingungen des Transports. Es fallen Frachtkosten an, und der Frachtführer haftet gemäß dem HGB für Verluste oder Beschädigungen der Güter. Die sorgfältige Planung und Durchführung von Transportgeschäften gewährleistet einen reibungslosen Warentransport. Das HGB bietet den rechtlichen Rahmen für den Transport von Gütern und legt die Rechte und Pflichten der Beteiligten fest.


\section{Beförderungsvertrag}
Ein Beförderungsvertrag ist ein rechtlicher Vertrag, der zwischen dem Beförderer und dem Versender abgeschlossen wird und den Transport von Gütern regelt. Es handelt sich um einen zentralen Bestandteil des Transportwesens, da er die rechtlichen Grundlagen für den Transport von Waren schafft und die Rechte und Pflichten der beteiligten Parteien festlegt.

Im Beförderungsvertrag werden verschiedene Elemente definiert. Zunächst werden die Parteien des Vertrags festgelegt. Der Beförderer ist die Person oder das Unternehmen, das den Transport der Güter durchführt, während der Versender die Partei ist, die die Güter zur Beförderung übergibt.

Der Beförderungsvertrag legt auch die Art der zu befördernden Güter fest. Dabei kann es sich um physische Güter wie Waren, Rohstoffe oder Maschinen handeln, aber auch um immaterielle Güter wie Daten oder Informationen, die elektronisch übertragen werden.

Ein weiterer wichtiger Bestandteil des Beförderungsvertrags ist die Festlegung des Transportwegs und des Transportmittels. Der Vertrag kann beispielsweise die Route von einem bestimmten Startpunkt zum Zielpunkt vorgeben und das gewählte Transportmittel spezifizieren, wie beispielsweise Straßentransport, Schifffahrt, Luftfracht oder Schienentransport.

Im Beförderungsvertrag werden auch die Bedingungen und Modalitäten des Transports geregelt.
\section{Haftung des Frachtführers für Schäden oder Verluste}
Der Frachtführer ist die Person oder das Unternehmen, das den Transport von Gütern im Rahmen eines Frachtvertrags durchführt. Gemäß dem HGB und anderen einschlägigen Gesetzen haftet der Frachtführer für Schäden, die während des Transports an den beförderten Gütern entstehen.

Die Haftung des Frachtführers basiert auf dem Prinzip der Verschuldenshaftung. Das bedeutet, dass der Frachtführer für Schäden oder Verluste haftet, die auf sein eigenes Verschulden, das Verschulden seiner Mitarbeiter oder auf Mängel in seiner Organisation oder Ausrüstung zurückzuführen sind. Die Haftung des Frachtführers besteht unabhängig davon, ob der Schaden durch Fahrlässigkeit, Vorsatz oder andere Umstände verursacht wurde.

Die Haftung des Frachtführers kann in verschiedene Phasen des Transports unterteilt werden:

\begin{itemize}
    \item \textbf{Vorannahme:} Der Frachtführer übernimmt die Güter und prüft ihre Beförderungstauglichkeit.
    \item \textbf{Beförderung:} Der Frachtführer ist für die sichere und termingerechte Lieferung der Güter verantwortlich.
    \item \textbf{Ablieferung:} Der Frachtführer übergibt die Güter an den Empfänger oder den berechtigten Empfangsberechtigten.
\end{itemize}

Die Haftung des Frachtführers erstreckt sich über den gesamten Transportprozess und umfasst die Verantwortung für Schäden oder Verluste, die auf sein Verschulden, das Verschulden seiner Mitarbeiter oder auf Mängel in seiner Ausrüstung oder Organisation zurückzuführen sind.


    % !TEX root =  master.tex
\chapter{Zusammenfassung}

Im Laufe der Arbeit wurde deutlich, dass das Handelsrecht im HGB eine bedeutende Rolle für die Regelung und den Schutz von Geschäften im Handelsverkehr spielt.

Persönlich finde ich diese Themen äußerst relevant und praxisnah. Das Verständnis der einzelnen Handelsgeschäfte nach dem HGB ermöglicht es den Akteuren im Handelsverkehr, ihre Rechte und Pflichten besser zu verstehen und angemessen zu handeln. Die Regelungen zum Speditions- und Frachtvertrag bieten eine klare rechtliche Grundlage für den Transport von Gütern und stellen sicher, dass sowohl der Spediteur als auch der Frachtführer bestimmten Haftungsbestimmungen unterliegen.

Die intensive Auseinandersetzung mit diesen Themen hat mir ein besseres Verständnis für die rechtlichen Aspekte des Handelsverkehrs vermittelt. Die Kenntnis dieser rechtlichen Bestimmungen ist von entscheidender Bedeutung, um potenzielle Risiken zu minimieren und rechtliche Konflikte zu vermeiden.

Die Hausarbeit hat mir gezeigt, dass das HGB ein unverzichtbares Instrument ist, um den Handelsverkehr zu regeln und einen reibungslosen Ablauf von Handelsgeschäften zu gewährleisten. Außerdem war es ebenfalls eine große Hilfe bei der korrekten Recherche in dieser Hausarbeit. Es ist von großer Bedeutung, dass Unternehmen und Einzelpersonen, die im Handelsverkehr tätig sind, sich mit den Bestimmungen des HGB vertraut machen und ihre Handelsaktivitäten entsprechend gestalten.

Insgesamt bin ich der Überzeugung, dass die Themen der einzelnen Handelsgeschäfte, des Speditions- und Frachtvertrags sowie der Lager- und Transportgeschäfte nach dem HGB eine solide Grundlage für den Handelsverkehr bilden. Sie bieten Rechtssicherheit, ermöglichen effiziente Geschäftsabläufe und schützen die Interessen der Vertragsparteien. Daher sollten Unternehmen und Einzelpersonen, die im Handelsverkehr tätig sind, sich intensiv mit diesen Themen auseinandersetzen, um von den rechtlichen Vorteilen und Schutzmechanismen des HGB zu profitieren.




%%%%%%%%%%%%%%%%%%%%%%%%%%%%%%%%%%%
% ANHÄNGE
%
% @stud: einzelne Anhänge bearbeiten und eigene Anhänge hier einfügen
%        die nachfolgenden Zeilen deaktivieren, wenn keine Anhänge verwendet werden
%
%\initializeAppendix

%%%%%%%%%%%%%%%%%%%%%%%%%%%%%%%%%%%

    \singlespacing

%%%%%%%%%%%%%%%%%%%%%%%%%%%%%%%%%%%
% LITERATURVERZEICHNIS
% @stud: Literaturverzeichnis in Datei bibliography.bib anpassen.
%
% Alternative zu Verwendung von \initializeBibliography: Citavi ...
% (dann \initializeBibliography auskommentieren und eigenes LaTex Coding verwenden)
%
    \initializeBibliography
%%%%%%%%%%%%%%%%%%%%%%%%%%%%%%%%%%%

\end{document}
